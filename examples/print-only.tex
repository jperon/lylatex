\documentclass{lyluatexexample}
\begin{document}

\def\postLilyPondExample{\par\bigskip\hrule\par\bigskip}

\section*{Print only Selected Systems or Pages}

The \texttt{print-only} and \texttt{do-not-print} options allow to limit
the printed systems or pages from a score. A typical use case is to print
a score interspersed with comments.  The advantage of this approach is that
the score is compiled only once while the individual systems are simply
reused by \LaTeX.

Throughout this document we'll demonstrate the different options to
select systems from the following score:

\lilypondfile[verbatim]{eight-systems.ly}

The simplest selection is a single system: \texttt{print-only=4}

\lilypondfile[print-only=4]{eight-systems.ly}

Ranges are also possible: \texttt{print-only=3-5}, with the special form of
\texttt{print-only=6-} which prints from the given system throughout the end of
the score. Negative ranges can be given with \texttt{print-only=7-5}

\lilypondfile[print-only=3-5]{eight-systems.ly}

\lilypondfile[print-only=6-]{eight-systems.ly}

\lilypondfile[print-only=7-5]{eight-systems.ly}

With a comma-separated list an arbitrary sequence of systems can be specified.
The list has to be enclosed in curly brackets: \texttt{print-only={4,1,2}}

\lilypondfile[print-only={4,1,2}]{eight-systems.ly}

Each element of the list can include any of the forms described above:\\
\texttt{print-only={3,5-7,4,7-}}

\lilypondfile[print-only={3,5-7,4,7-}]{eight-systems.ly}

\texttt{do-not-print} does the opposite: it prevents the list of systems from
being printed. It might be used alone, or in combination with
\texttt{print-only}:\\
\texttt{print-only=3-,do-not-print=6}

\lilypondfile[print-only=3-,do-not-print=6]{eight-systems.ly}

The functionality is identical with fullpage scores where the selection applies
to \emph{pages} instead. This can for example be used when the “score” file
contains a number of individual pieces (e.g. songs for a song book), and
individual selections are to be printed.

Systems have some specific behaviour with regard to \emph{indent},
but this is demonstrated in its own file \texttt{dynamic-indent.tex}.

\let\postLilyPondExample\undefined

\end{document}
